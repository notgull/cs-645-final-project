\documentclass[12pt]{article}

%
%Margin - 1 inch on all sides
%
\usepackage[letterpaper]{geometry}
\usepackage{times}
\geometry{top=1.0in, bottom=1.0in, left=1.0in, right=1.0in}

%
%Doublespacing
%
\usepackage{setspace}
\doublespacing

%
%Rotating tables (e.g. sideways when too long)
%
\usepackage{rotating}


%
%Fancy-header package to modify header/page numbering (insert last name)
%
\usepackage{fancyhdr}
\pagestyle{fancy}
\lhead{May 8, 2023} 
\chead{\textbf{Emulating an IPS over an IDS using Iptables}} 
\rhead{John Nunley \thepage} 
\lfoot{} 
\cfoot{} 
\rfoot{} 
\renewcommand{\headrulewidth}{0pt} 
\renewcommand{\footrulewidth}{0pt} 
%To make sure we actually have header 0.5in away from top edge
%12pt is one-sixth of an inch. Subtract this from 0.5in to get headsep value
\setlength\headsep{0.333in}

%
%Works cited environment
%(to start, use \begin{workscited...}, each entry preceded by \bibent)
% - from Ryan Alcock's MLA style file
%
\newcommand{\bibent}{\noindent \hangindent 40pt}
\newenvironment{workscited}{\newpage \textbf{7). References}}{\newpage }


%
%Begin document
%
\begin{document}
\begin{flushleft}

%%%%Changes paragraph indentation to 0.5in
%\setlength{\parindent}{0.5in}
%%%%Begin body of paper here

\textbf{1). Abstract}

TODO:

- Hosts may spoof their IP addresses

\textbf{2). Introduction}

For ensuring the continued integrity of computer networks, it is of the utmost importance to make sure no intrusions occur. Intrusions are defined as "A security event, or a combination of multiple security events, that constitutes a security incident in which an intruder gains, or attempts to gain, access to a system or system resource without having authorization to do so" (NIST). 

\textbf{3). Challenges in the security issue/threat model}

\textbf{4). Existing research}

\textbf{5). Open Research Challenges/Solution}

\textbf{6). Conclusions}

\newpage

%%%%Works cited
\begin{workscited}

https://csrc.nist.gov/glossary/term/intrusion

\bibent
Allen, R.L. \textit{The American Farm Book; or Compend of Ameri can Agriculture; Being a Practical Treatise on Soils, Manures, Draining, Irrigation, Grasses, Grain, Roots, Fruits, Cotton, Tobacco, Sugar Cane, Rice, and Every Staple Product of the United States with the Best Methods of Planting, Cultivating, and Prep aration for Market.} New York: Saxton, 1849. Print.

\bibent
Baker, Gladys L., Wayne D. Rasmussen, Vivian Wiser, and Jane M. Porter. \textit{Century of Service: The First 100 Years of the United States Department of Agriculture.}[Federal Government], 1996. Print.

\bibent
Danhof, Clarence H. \textit{Change in Agriculture: The Northern United States, 1820-1870.} Cambridge: Harvard UP, 1969. Print.


\end{workscited}

\end{flushleft}
\end{document}
\}